
%%%%%%%%%%%%%%%%%%%%%%%%%%%%%%%%%%%%%%%%%
% University/School Laboratory Report
% LaTeX Template
% Version 3.0 (4/2/13)
%
% This template has been downloaded from:
% http://www.LaTeXTemplates.com
%
% Original author:
% Linux and Unix Users Group at Virginia Tech Wiki 
% (https://vtluug.org/wiki/Example_LaTeX_chem_lab_report)
%
% License:
% CC BY-NC-SA 3.0 (http://creativecommons.org/licenses/by-nc-sa/3.0/)
%
%%%%%%%%%%%%%%%%%%%%%%%%%%%%%%%%%%%%%%%%%

%----------------------------------------------------------------------------------------
%	PACKAGES AND DOCUMENT CONFIGURATIONS
%----------------------------------------------------------------------------------------



\documentclass[11pt,a4paper,polish]{article}
\usepackage[T1]{fontenc}
\usepackage[utf8]{inputenc}
\usepackage{babel}
\usepackage{blindtext}

\linespread{1.3} %interlinia
\addtolength{\textwidth}{3cm}
\addtolength{\hoffset}{-1.5cm}
\addtolength{\textheight}{3cm}
\addtolength{\voffset}{-1.5cm}

\usepackage{graphicx} % Required for the inclusion of images


%----------------------------------------------------------------------------------------
%	DOCUMENT INFORMATION
%----------------------------------------------------------------------------------------

\title{Algorytmy rotacyjne - symulacja} % Title

\author{Łukasz Sędek, Marcin Toczko\\ \{LSedek, MToczko \} 
@stud.elka.pw.edu.pl}
% Author name

\date{\today} % Date for the report

\begin{document}

\maketitle % Insert the title, author and date



% If you wish to include an abstract, uncomment the lines below
% \begin{abstract}
% Abstract text
% \end{abstract}

%----------------------------------------------------------------------------------------
%	WPROWADZENIE
%----------------------------------------------------------------------------------------

\section{Wprowadzenie}

Niniejsza praca jest projektem na przedmiot Metody Kryptografii i Ochrony
Informacji. Opracowane w niej zostało działanie algorytów rotracyjnych na
podstawie symulacji szyfratora Enigma.


\subsection{Definitions}
\label{definitions}
\begin{description}
\item[Stoichiometry]
The relationship between the relative quantities of substances taking part in a reaction or forming a compound, typically a ratio of whole integers.
\item[Atomic mass]
The mass of an atom of a chemical element expressed in atomic mass units. It is approximately equivalent to the number of protons and neutrons in the atom (the mass number) or to the average number allowing for the relative abundances of different isotopes. 
\end{description} 
 
%----------------------------------------------------------------------------------------
%	SECTION 2
%----------------------------------------------------------------------------------------

\section{Opracowanie teoretyczne}

Mechanizm rotacyjny wykorzystywany jest przez szyfry strumieniowe. Zasada
działania szyfrów strumieniowych opiera na szyfrowaniu jawnego tekstu ze 
znanego alfabetu (uporządkowanego zbioru możliwych danych wejściowych) i klucza
bieżącego. Kluczem bieżącym może być dowolna fraza o długości nie mniejszej
od długości tekstu szyfrowanego (jawnego)[1]. Jednak użycie tekstu w roli klucza bieżącego ( fragmentu książki ) stwarza możliwość odczytania wiadomości
wykorzystując redundancję języka. Zalecane jest stosowanie losowego ciągu nie
przenoszącego żadnej logicznej informacji, zarazem dany ciąg losowy powinien
być wykorzystany tylko jeden raz. Wtórne użycie danego ciągu, zwiększa prawdopodobieństwa odczytania wiadomości.
Zilustrowaną zasadę działania rotorów przedstawiono na rysunku 1. Podstawą
działania są niezależnie obracające się cylinry, które przenoszą syngał elektryczny. Rysunek jest przedstawia realizację Enigmy. Każdy cylinder jest wyposażony
w 26 styków wejściowych i 26 wyjściowych oraz wewnętrzne przewody łączące
każdy styk wejściowy z odpowiadającym mu wyjściowym. Złożoność, jak i siła
szyfrowania tkwi w liczbie cylindrów i liczbie permutacji jakie mogą nastąpić.
Proces odwrotny do szyfrowania jest możliwy jedynie , gdy układ cylindrów
po stronie nadawczej oraz odbiorczej jest taki sam. Kluczem algorytmicznym
jest identyczne ustawienie początkowe walców w maszynach. Podczas II wojny światowej Enigma była wykorzystywana jako główne narzędzie szyfrujące w
rękach III Rzeszy. Siła takiego rozwiązania polegała, na częstej zmianie kluczy
szyfrujących (klucze dzienne). Podstawą działania była operacja podstawienia,
wchodząca litera była zamian zgodnie z ustawieniem stykami w danym rotorze
na inną. Złożoność systemu polegała na wielokrotnym użyciu operacji podstawienia. Po każdorazowym naciśnięciu klawisza następowała zmiana ustawień
rotorów.
Zaletą takich algorytmów jest duży okres powtarzalności klucza. Dla szyfratora
z alfabetem 26 znakowym i liczbą cylindrów k, klucz zostanie powtórzony co
26k obrotów najwolniejszego cylindra[2].

%----------------------------------------------------------------------------------------
%	SECTION 3
%----------------------------------------------------------------------------------------

\section{Koncepcja programu i idea algorytmu}

%----------------------------------------------------------------------------------------
%	SECTION 4
%----------------------------------------------------------------------------------------

\section{Results and Conclusions}


%----------------------------------------------------------------------------------------
%	SECTION 5
%----------------------------------------------------------------------------------------

\section{Discussion of Experimental Uncertainty}


The most obvious source of experimental uncertainty is the limited precision of the balance. Other potential sources of experimental uncertainty are: the reaction might not be complete; if not enough time was allowed for total oxidation, less than complete oxidation of the magnesium might have, in part, reacted with nitrogen in the air (incorrect reaction); the magnesium oxide might have absorbed water from the air, and thus weigh ``too much." Because the result obtained is close to the accepted value it is possible that some of these experimental uncertainties have fortuitously cancelled one another.

%----------------------------------------------------------------------------------------
%	Testy
%----------------------------------------------------------------------------------------

\section{Testy}

\begin{enumerate}
\begin{item}
The \emph{atomic weight of an element} is the relative weight of one of its atoms compared to C-12 with a weight of 12.0000000$\ldots$, hydrogen with a weight of 1.008, to oxygen with a weight of 16.00. Atomic weight is also the average weight of all the atoms of that element as they occur in nature.
\end{item}
\begin{item}
The \emph{units of atomic weight} are two-fold, with an identical numerical value. They are g/mole of atoms (or just g/mol) or amu/atom.
\end{item}
\begin{item}
\emph{Percentage discrepancy} between an accepted (literature) value and an experimental value is $\frac{|\mathrm{experimental result} - \mathrm{accepted result}|}{\mathrm{accepted result}}$.
\end{item}
\end{enumerate}

%----------------------------------------------------------------------------------------
%	BIBLIOGRAPHY
%----------------------------------------------------------------------------------------

\bibliographystyle{unsrt}

\bibliography{sample}

%----------------------------------------------------------------------------------------


\end{document}